\documentclass[a4paper,11pt]{article}
\usepackage{enumitem}
\usepackage{amssymb}
\usepackage{amsmath}

\title{Analysis Übungsblatt 01}

\begin{document}
\maketitle

\section{Aufgabe 01: Dreiecksungleichung}
Es gilt $\forall x,y\in\mathbb{R}$ die \emph{Dreiecksungleichung}: $|x+y|\leq|x|+|y|$.\\
Zeigen sie mittels \emph{vollständiger Induktion} folgende vereinfachte Gleichung:\\
$|\underset{k=1}{\overset{n}{\sum}}a_k|\leq\underset{k=1}{\overset{n}{\sum}}|a_k|\forall n\in\mathbb{N},a_1,\cdots,a_n\in\mathbb{R}$ (IH)\\
\underline{Lösung}:\\
(IA) $n=1: |\underset{k=1}{\overset{1}{\sum}}a_k|=|a_1|\leq|a_1|=\underset{k=1}{\overset{1}{\sum}}|a_k|$\\
(IS) $n=n+1:|\underset{k=1}{\overset{n+1}{\sum}}a_k|=|\underset{k=1}{\overset{n}{\sum}}a_k+a_{n+1}|\overset{trivial}{=}|\underset{k=1}{\overset{n}{\sum}}a_k|+|a_{n+1}|\\
\overset{IH}{\leq}\underset{k=1}{\overset{n}{\sum}}|a_k|+|a_{n+1}|=\underset{k=1}{\overset{n+1}{\sum}}|a_k|$
\section{Aufgabe 02: Geometrische Summe}
\begin{enumerate}[label={\alph*)}]
	\item Beweisen sie mittel \emph{vollständiger Induktion}:\\
		$n\in\mathbb{N}_0,q\neq1:\underset{k=0}{\overset{n}{\sum}}q^k=\frac{1-q^{n+1}}{1-q}$ (IH)\\
		(IA) $n=0:\underset{k=0}{\overset{0}{\sum}}q^0=q^0=1=\frac{1-q}{1-q}=\frac{1-q^{0+1}}{1-q}$\\
		(IS) $n=n+1:\underset{k=0}{\overset{n+1}{\sum}}q^k=\underset{k=0}{\overset{n}{\sum}}q^k+q^{n+1}\overset{IH}{=}\frac{1-q^{n+1}}{1-q}+q^{n+1}\\
		=\frac{1-q^{n+1}}{1-q}+\frac{q^{n+1}}{1}=\frac{1(1-q^{n+1})}{1(1-q)}+\frac{(1-q)(q^{n+1})}{1(1-q)}\\
		=\frac{1-q^{n+1}}{1-q}+\frac{q^{n+1}-q^{n+2}}{1-q}=\frac{1-q^{n+1}+q^{n+1}-q^{n+2}}{1-q}=\frac{1-q^{n+2}}{1-q}=\frac{1-q^{(n+1)+1}}{1-q}$
	\item Die beschriebene Summe verhält sich für $a(1)=1, a(2)=2, a(3)=4$ usw., d.h. es gilt folgende \emph{geometrische} Folge: $a(n)=2^{n-1}$\\
		Für die Lösung müssen wir lediglich folgende Formel ausrechnen:

		$\underset{k=1}{\overset{64}{\sum}}2^{k-1}\overset{Indexversch.}{=}\underset{k=0}{\overset{63}{\sum}}2^k\overset{geom. SF}{=}\frac{1-2^{63+1}}{1-2}=\frac{1-2^{64}}{-1}=\frac{1}{-1}-\frac{2^{64}}{-1}= -1+\frac{2^{64}}{1}=2^{64}-1=18446744073709551615$

		Man bräuchte als genau 18446744073709551615 Reiskörner.
\end{enumerate}
\section{Aufgabe 03: Eigenschaften von Folgen}
Geben sie die \emph{nächsten 4 Glieder} an, das \emph{Bildungsgesetz} und die \emph{Eigenschaften} der Folgen.
\begin{enumerate}[label={\alph*)}]
	\item $1,3,5,7,\emph{9,11,13,15}$

		$a(n)=(2n)-1$ (arithmetische, streng monoton wachsend)
	\item $1,-2,4,-8,\emph{16, -32, 64, -128}$

		$a(n)=(-1)^{n-1}2^{n-1}$ (geometrisch, alternierend)
	\item $1,1,1,1,\emph{1,1,1,1}$

		$a(n)=1$ (monoton wachsend und fallend,\\
		nach oben und unten beschränkt für $c>1$)
	\item $\frac{1}{2},\frac{3}{4},\frac{7}{8},\frac{15}{16},\frac{31}{32},\frac{63}{64},\frac{127}{128},\frac{255}{256}$

		$a(n)=\frac{2n-1}{2n}$ (nach oben beschränkt für $c<1$)
	\item $2,3,5,7,11,13,17,15,17,19,21$

		$a(n)=\begin{Bmatrix}n=1:2\\
			n=5,6: 2n+1\\
			n=7: 2n+3\\
			\text{ansonsten: } 2n-1
		      \end{Bmatrix}$
	\item $1,1,2,3,5,8,13,21,34,55,89,144$

		$a(n)=\begin{Bmatrix}n=0,1: 1\\
			\text{ansonsten}: a_{n-1}+a_{n-2}
		\end{Bmatrix}$

		Fibbonachi-Folge ist streng monoton wachsend und arithmetisch.
	\item $1,1.4,1.41,1.414,1.4142,$

		tba
\end{enumerate}
\section{Aufgabe 04: Folgen mit bestimmten Eigenschaften}
Geben sie wenn möglich eine Beispielfolge an:
\begin{enumerate}[label={\alph*)}]
	\item geometrisch, streng monoton wachsend:

		tba
	\item geometrisch, streng monoton fallend:

		tba
	\item arithmetisch, alternierend:

		tba
	\item arithmetisch, nach oben beschränkt:

		tba
	\item arithmetisch, beschränkt:

		tba
\end{enumerate}
\end{document}
