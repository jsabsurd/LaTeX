\documentclass[a4paper,11pt]{article}
\usepackage{enumitem}
\usepackage{amssymb}
\usepackage{amsmath}

\title{Analysis Übungsblatt 01}

\begin{document}
\maketitle

\section{Aufgabe 01: Dreiecksungleichung}
Es gilt $\forall x,y\in\mathbb{R}$ die \emph{Dreiecksungleichung}: $|x+y|\leq|x|+|y|$.\\
Zeigen sie mittels \emph{vollständiger Induktion} folgende vereinfachte Gleichung:\\
$|\underset{k=1}{\overset{n}{\sum}}a_k|\leq\underset{k=1}{\overset{n}{\sum}}|a_k|\forall n\in\mathbb{N},a_1,\cdots,a_n\in\mathbb{R}$ (IH)\\
\underline{Lösung}:\\
(IA) $n=1: |\underset{k=1}{\overset{1}{\sum}}a_k|=|a_1|\leq|a_1|=\underset{k=1}{\overset{1}{\sum}}|a_k|$\\
(IS) $n=n+1:|\underset{k=1}{\overset{n+1}{\sum}}a_k|=|\underset{k=1}{\overset{n}{\sum}}a_k+a_{n+1}|\overset{trivial}{=}|\underset{k=1}{\overset{n}{\sum}}a_k|+|a_{n+1}|\\
\overset{IH}{\leq}\underset{k=1}{\overset{n}{\sum}}|a_k|+|a_{n+1}|=\underset{k=1}{\overset{n+1}{\sum}}|a_k|$
\section{Aufgabe 02: Geometrische Summe:}
\begin{enumerate}[label={\alph*)}]
	\item Beweisen sie mittel \emph{vollständiger Induktion}:\\
		$n\in\mathbb{N}_0,q\neq1:\underset{k=0}{\overset{n}{\sum}}q^k=\frac{1-q^{n+1}}{1-q}$ (IH)\\
		(IA) $n=0:\underset{k=0}{\overset{0}{\sum}}q^0=q^0=1=\frac{1-q}{1-q}=\frac{1-q^{0+1}}{1-q}$\\
		(IS) $n=n+1:\underset{k=0}{\overset{n+1}{\sum}}q^k=\underset{k=0}{\overset{n}{\sum}}q^k+q^{n+1}\overset{IH}{=}\frac{1-q^{n+1}}{1-q}+q^{n+1}\\
		=\frac{1-q^{n+1}}{1-q}+\frac{q^{n+1}}{1}=\frac{1(1-q^{n+1})}{1(1-q)}+\frac{(1-q)(q^{n+1})}{1(1-q)}\\
		=\frac{1-q^{n+1}}{1-q}+\frac{q^{n+1}-q^{n+2}}{1-q}=\frac{1-q^{n+1}+q^{n+1}-q^{n+2}}{1-q}=\frac{1-q^{n+2}}{1-q}=\frac{1-q^{(n+1)+1}}{1-q}$
	\item Die beschriebene Summe verhält sich für $a(1)=1, a(2)=2, a(3)=4$ usw., d.h. es gilt folgende \emph{geometrische} Folge: $a(n)=2^{n-1}$\\
		Für die Lösung müssen wir lediglich folgende Formel ausrechnen\\
		$\underset{k=1}{\overset{64}{\sum}}2^{k-1}\overset{Indexversch.}{=}\underset{k=0}{\overset{63}{\sum}}2^k$

		tba.
\end{enumerate}
\section{Aufgabe 03}
\section{Aufgabe 04}
\end{document}
