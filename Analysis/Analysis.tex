\documentclass[a4paper,11pt]{article}

\usepackage{enumitem}
\usepackage{amssymb}
\usepackage{amsmath}

\title{Überarbeitete Version d. Analysis Mitschriebes}

\begin{document}
\maketitle
\tableofcontents

\section{Grundlagen}
\subsection{Def.: Grundlegendes}
\begin{enumerate}[label={\alph*)}]
	\item Eine \underline{Definition} führt einen neuen Begriff oder eine neue Schreibweise ein, und erklärt ihre Bedeutung unter Verwendung bereits bekannter Begriffe und Schreibweisen.\\
	Der neue Begriff wird zur Verdeutlichung unterstrichen.
	\item Ein \underline{Satz} formuliert eine wahre Aussage, z.b. als \emph{wenn-dann-Aussage}.
	\item Ein \underline{Beweis} zeigt die Richtigkeit eines Satzes, meist unter Verwendung bereits bewiesener Sätze.
\end{enumerate}
\subsection{Def.: Natürliche Zahlen}
\begin{enumerate}[label={\alph*)}]
	\item Die Elemente der Menge \( \mathbb{N} \) heißen \textbf{natürliche Zahlen}.
	\item \(\mathbb{N}_{0}:=\mathbb{N} \cup 0\)
\end{enumerate}
\subsection{Satz: Vollständige Induktion}
Gilt eine Aussage $A(n)$ für $n=1$ (IA),\\
und man beweißt für $n=n+1$ (IS),\\
so gilt die Induktionshypothese (IH)\\
mittels \underline{Vollständiger Induktion} als bewiesen.
\subsection{Satz: Gaußsche Summenformel}
\begin{equation} \label{gauß1}
	\forall n\in\mathbb{N}:\underset{k=1}{\overset{n}{\sum}}k=1+2+\cdots+n:=\frac{n(n+1)}{2}
\end{equation}
\emph{Beweis durch vollst. Induktion:}\\
IA: $n=1:\underset{k=1}{\overset{1}{\sum}}k=1=\frac{1(1+1)}{2}$\\
IS: $n\rightarrow n+1:\overset{n+1}{\underset{k=1}{\sum}}k=\overset{n}{\underset{k=1}{\sum}}k+(n+1)\overset{IH}{=}\frac{n(n+1)}{2}+(n+1)$\\
$=\frac{n(n+1)}{2}+\frac{(n+1)}{1}=\frac{n(n+1)}{2}+\frac{2(n+1)}{2}=\frac{n(n+1)}{2}+\frac{2n+2}{2}$\\
$=\frac{(n(n+1))+(2n+2)}{2}=\frac{(n^2+n)+(2n+2)}{2}=\frac{n^2+n+2n+2}{2}$\\
$=\frac{(n+1)(n+2)}{2}=\frac{(n+1)((n+1)+1)}{2}$
\subsection{Def.: Funktion}
\begin{enumerate}[label={\alph*)}]
	\item Eine \underline{Funktion} $f:D\to W$ ordnet jedem Wert $x\in D$ einen Wert $f(x)\in W$ zu.
	\item $D$ heißt \underline{Definitionsmenge}, $W$ heißt \underline{Wertemenge} der Funktion $f$.
\end{enumerate}
\section{Folgen}
\subsection{Def.: Folgen}
\begin{enumerate}[label={\alph*)}]
	\item Eine \underline{Folge} $(a_n)$ ist eine Funktion $\mathbb{N}\rightarrow\mathbb{R}$,\\
		die jeder natürlichen Zahl $n$ eine reelle Zahl $a_n$ zuordnet.\\
		(Bezeichnung der Glieder: $a_1,a_2,...,a_{n-1},a_n,a_{n+1}$)
	\item Spezielle Folgen:
		\begin{itemize}
			\item \emph{arithmetische Folgen}:\\
				$\forall n\in \mathbb{N}:a_{n+1}=a_n+d$ für ein bel. aber festes $d\in\mathbb{R}$.\\
				z.b.: $a_n=2n$, $a_n=5-3n$
			\item \emph{geometrische Folgen}:\\
				$\forall n\in \mathbb{N}:a_{n+1}=a_n*q$ für ein bel. aber festes $q\in\mathbb{R}$.\\
				z.b.: $a_n=2^n$, $a_n=\frac{1}{3n}$
			\item \emph{alternierende Folgen}:\\
				$\forall n\in\mathbb{N}:a_{n+1}a_n<0$
			\item \emph{(streng) monoton wachsende Folgen}:\\
				$\forall n\in\mathbb{N}:a_{n+1}\geq(,>)a_n$
			\item \emph{(streng) monoton fallende Folgen}:\\
				$\forall n\in\mathbb{N}:a_{n+1}\leq(,<)a_n$
			\item \emph{(nach oben/unten) beschränkte Folgen}:\\
				$\forall n\in\mathbb{N}:|a_n|\leq(,\geq) c$ mit $c\in\mathbb{R}$
		\end{itemize}
\end{enumerate}
\subsection{Def.: Grenzwerte}
	Eine Folge $(a_n)$ besitzt den \underline{Grenzwert} (GW) $a\in\mathbb{R}$, geschrieben
	\[\lim_{x\to\infty}a_n=a \text{ bzw. } a_n\underset{{n\to\infty}}{\to}a\]
	Wenn $\forall \epsilon>0\exists N(\epsilon)>0:|a_n-a|<\epsilon, \forall n>N(\epsilon)$
\subsection{Def.: Eigenschaften von Grenzwerten}
\begin{enumerate}[label={\alph*)}]
	\item \underline{konvergent} $\Leftrightarrow\exists a\in\mathbb{R}$
		\begin{itemize}
			\item \underline{Nullfolge} $\Leftrightarrow a = 0$
		\end{itemize}
	\item \underline{divergent} $\Leftrightarrow\nexists a\in\mathbb{R}$
		\begin{itemize}
			\item \underline{bestimmt divergent} $\Leftrightarrow\exists a\text{ mit }a=\begin{Bmatrix}\infty\\-\infty\end{Bmatrix}$
			\item \underline{unbestimmt divergent} $\Leftrightarrow\nexists a$
		\end{itemize}
\end{enumerate}
\subsection{Def.: Uneigentlicher Grenzwert}
	Eine Folge heißt \underline{bestimmt divergent} oder \underline{uneigentlich konvergent}, wenn sie als \underline{uneigentlichen GW}
		$\begin{Bmatrix}
			\infty\\
			-\infty
		\end{Bmatrix}$ besitzt.\\
		Geschrieben:
		\[\begin{Bmatrix}
			\underset{n\rightarrow\infty}{\lim}a_n=\infty\text{ oder }a_n\underset{n\to\infty}{\to}\infty\\
			\underset{n\rightarrow\infty}{\lim}a_n=-\infty\text{ oder }a_n\underset{n\to\infty}{\rightarrow}-\infty
		\end{Bmatrix}\]
		d.h. $\forall\begin{Bmatrix}A>0\\A<0\end{Bmatrix}\exists N(A)>0: \begin{Bmatrix}a_n>A\\a_n<A\end{Bmatrix}\forall n\geq N(A)$
\subsection{Satz: Eindeutigkeit}
Wenn eine Folge einen GW besitzt, ist sie \underline{eindeutig}.
\subsection{Satz: Rechenregeln für GW}
Für konvergente Folgen $a_n\underset{n\to\infty}{\rightarrow}a$ und $b_n\underset{n\to\infty}{\to}b$ gilt:
\begin{enumerate}[label={\alph*)}]
	\item $a_n+b_n\underset{n\to\infty}{\to}a+b$
	\item $a_n-b_n\underset{n\to\infty}{\to}a-b$
	\item $a_nb_n\underset{n\to\infty}{\to}ab$
	\item $\frac{a_n}{b_n}\underset{n\to\infty}{\to}\frac{a}{b}$ wenn $b_n\neq 0\forall n\in\mathbb{N}$
	\item $|a_n|\underset{n\to\infty}{\to}|a|$
	\item $\forall n\geq n_0:a_n\leq b_n\Rightarrow a\leq b$
\end{enumerate}
\section{Beispiele}
\subsection{Grenzwertnachweise}
\begin{enumerate}[label={\alph*)}]
	\item $a_n=\frac{2_n+1}{3n}$, Beh.: $\underset{n\to\infty}{\lim}a_n=\frac{2}{3}$\\
		\underline{Beweis:} sei $\epsilon=0$ bel. aber fest.\\
		$|a_n-a|=|\frac{2n+1}{3n}-\frac{2}{3}|=|\frac{2n+1-2n}{3n}|=|\frac{1}{3n}|=\frac{1}{3n}\stackrel{!?}{<}\epsilon$\\
		$\frac{1}{3}<\epsilon n\Leftrightarrow \frac{1}{3\epsilon}<n$\\
		Wählen wir für $N(\epsilon)=\frac{1}{3\epsilon}$, gilt $\forall n>N(\epsilon):|a_n+a|<\epsilon$
	\item $a_n=\frac{2n+1}{3n+4}$, Beh.: $\underset{n\to\infty}{\lim}a_n=\frac{2}{3}$\\
		\underline{Beweis}: sei $\epsilon>0$ bel. aber fest\\
		$|a_n-a|=|\frac{2n+1}{3n+4}-\frac{2}{3}|=|\frac{3(2n+1)}{3(3n+4)}-\frac{2(3n+4)}{3(3n+4)}|=|\frac{3(2n+1)-2(3n+4)}{3(3n+4)}|=|\frac{6n+3-6n-8}{9n+12}|=|\frac{-5}{9n+12}|=\frac{5}{9n+12}\overset{!?}{<}\epsilon$\\
		$\Leftrightarrow 5<\epsilon(9n+12) \Leftrightarrow\frac{5}{\epsilon}<9n+12\Leftrightarrow\frac{5}{\epsilon}-12<9n\Leftrightarrow\frac{\frac{5}{\epsilon}-12}{9}<n\Leftrightarrow n>\frac{\frac{5}{\epsilon}-12}{9}$\\
		Deshalb wählen wir für $N(\epsilon)=\frac{\frac{5}{\epsilon}-12}{9}$, sodass gilt:\\
		$\forall n>N(\epsilon):|a_n-a|<\epsilon$\\
		\emph{Alternativ}: $|a_n-a|=...=\frac{5}{9n+12}<\frac{5}{9n}<\frac{1}{n}\overset{!?}{<}\epsilon$\\
		d.h. $n>\frac{1}{\epsilon}$, wähle daher $N(\epsilon)=\frac{1}{\epsilon}$
	\item $a_n=\frac{2n+1}{3n-4}$, Beh.: $a_n\underset{n\to\infty}{\lim}\frac{2}{3}$\\
		\underline{Beweis:} sei $\epsilon>0$ bel. aber fest\\
		$|a_n-a|=|\frac{2n+1}{3n-4}-\frac{2}{3}|=|\frac{3(2n+1)}{3(3n-4)}-\frac{2(3n-4)}{3(3n-4)}|=|\frac{(3(2n+1))-(2(3n-4))}{3(3n-4)}|=|\frac{6n+3-(6n-8)}{9n-12}|=|\frac{11}{9n-12}|$\\
		$\overset{\forall n\geq2}{=}\frac{11}{9n-12}=\frac{11}{8n+n-12}\overset{\forall n\geq12}{<}\frac{11}{8n}<\frac{2}{n}\overset{!?}{<}\epsilon$ wenn $n>\frac{2}{\epsilon}$\\
		Deshalb wählen wir $N(\epsilon)=max(\frac{2}{\epsilon},12)$, sodass gilt:\\
		$n>N(\epsilon):|a_n-a|<\epsilon$
	\end{enumerate}
\subsection{Uneigentlichen Grenzwert nachweisen}
$a_n=\frac{2n^2+1}{3n+4}$, Beh.: $\underset{n\to\infty}{\lim}a_n=\infty$\\
\emph{Beweis:} sei $A>0$ bel. aber fest.\\
$a_n=\frac{2n^2+1}{3n+4}>\frac{2n^2}{3n+4}=\frac{2n^2}{2n^2}{4n-(n-4)}\overset{\forall n>4}{>}\frac{2n^2}{4n}=\frac{n}{2}\overset{!?}{>}A$, wenn $n>2A$\\
Wähle daher $N(A)=max(2A, 4): n\geq N(A):a_n>A$
\subsection{Rechenregeln für Grenzwerte}
$a_n=\frac{2n+1}{3n-4}=\frac{2n(2+\frac{1}{n})}{n(3-\frac{4}{n})}=\frac{2+\frac{1}{n}}{3-\frac{4}{n}}\overset{2.7}{\underset{n\to\infty}{\to}}\frac{2+0}{3-0}=\frac{2}{3}$
\subsection{diverse}
\begin{enumerate}[label={\alph*)}]
	\item $a_n=\frac{3n^3-4n^2+7n-1}{-5n^3+2n^2-8n}=\frac{n^9(3-\frac{4}{n}+\frac{7}{n^2}-\frac{1}{n^3})}{n^3(-5+\frac{2}{n}-\frac{1}{n^3})}$\\
		$=\frac{3-\frac{4}{n}+\frac{7}{n^2}+\frac{1}{n^3}}{-5+\frac{2}{n}-\frac{8}{n^2}}\underset{n\to\infty}{\to}\frac{3-0+0+0}{-5+0-0}=-\frac{3}{5}$
	\item $a_n=\frac{2n+1}{3n-4}$, Beh.: $a_n\underset{n\to\infty}{\to}\frac{7}{3}$\\
		\emph{Beweis:} sei $\epsilon$ bel. aber fest.\\
		Vorüberlegung zur Wahl von $N(\epsilon)$:\\
		$|a_n-a|=|\frac{2n+1}{3n-4}-\frac{1}{3}|=|\frac{3(2n+1)-2(3n-4)}{3(3n-4)}|=|\frac{6n+3-6n+8}{9n-12}|=|\frac{11}{9n-12}|\overset{n=2}{=}\frac{11}{9n-12}=\frac{11}{8n+(n-12)}\leq\frac{11}{8n}<\frac{2}{n}\overset{!?}{<}\epsilon$, für $n>\frac{2}{\epsilon}$\\
		Wähle daher $N(\epsilon)=max(\frac{2}{\epsilon},12)$.
		Dann gilt für $n<N(\epsilon):|a_n-a|<\epsilon$
	\item $a_n=\frac{3n^3-4n^2+7n-1}{-5n^4+2n^2-8n}=\frac{n^3(3-\frac{4}{n}+\frac{7}{n^2}-\frac{1}{n^3})}{n^4(-5+\frac{2}{n^2}-\frac{8}{n^3})}$\\
		$=\frac{1}{n}\frac{3-\frac{4}{n}+\frac{7}{n^2}-\frac{1}{n^3}}{-5+\frac{2}{n^2}-\frac{8}{n^3}}\underset{n\to\infty}{\to}\frac{3-0+0-0}{-5+0-0}=0$
	\item $a_n=\frac{3n^4-4n^2+7n-1}{-5n^3+2n^2-8n}=\frac{n^4(3-\frac{4}{n^2}+\frac{7}{n^3}-\frac{1}{n^4})}{n^3(-5+\frac{2}{n}-\frac{8}{n^4})}=\frac{3n-\frac{4}{n^2}+\frac{7}{n^3}-\frac{1}{n^4}}{-5+\frac{7}{n}-\frac{8}{n^2}}$\\
		$\underset{n\to\infty}{\to}\frac{\infty*3-0+0-0}{-5+0-0}=\infty\frac{3}{-5}=-\infty$
\end{enumerate}
\end{document}
